\documentclass[11pt,letterpaper]{scrartcl}


%%%%%%%%%%%%%%%%%%%%%%%%%%%%%%%
% CHARACTER ENCODING PACKAGES %
%%%%%%%%%%%%%%%%%%%%%%%%%%%%%%%

\usepackage[utf8]{inputenc}
\usepackage[T1]{fontenc}
\usepackage[english]{babel}

\DeclareUnicodeCharacter{00A0}{ } % To avoid no-break space problems in utf8

%%%%%%%%%%%%%%%%%%%%%%%%%%%%%%
% PAGE LAYOUT SETUP PACKAGES %
%%%%%%%%%%%%%%%%%%%%%%%%%%%%%%

\usepackage{layout}
\usepackage{setspace}
%\usepackage[top=2cm, bottom=2cm, left=1.5cm, right=1.5cm]{geometry} % filled pages
\usepackage[top=3cm, bottom=3cm, left=2cm, right=2cm]{geometry}
%\usepackage{fancyhdr}
%\usepackage{ragged2e}

\usepackage[bottom]{footmisc} % Footnotes always at the bottom, even with floats

% \fancyhf{}
% \fancyhead[LE,RO]{\chaptertitle}
% \fancyhead[RE,LO]{}
% \fancyfoot[CE,CO]{}
% \fancyfoot[LE,RO]{\thepage}

% \newpagestyle{newstyle}{
%   \setheadrule{0pt}% Header rule
%   \sethead[\textit{\chaptertitle}]% even left
%     []% even centre
%     []% even right
%     {}% odd left
%     {}% odd centre
%     {\textit{\chaptertitle}}% odd right
%   \setfoot[]% even left
%     [\thepage]% even centre
%     []% even right
%     {}% odd left
%     {\thepage}% odd centre
%     {}% odd right
% }

\pagestyle{plain}

%\setlength{\parindent}{4em} 	% Paragraph indentation space setting
%\setlength{\parskip}{1em}		% Spacing between paragraphs

% Footnotes without markers :
\newcommand\blfootnote[1]{%
  \begingroup
  \renewcommand\thefootnote{}\footnote{#1}%
  \addtocounter{footnote}{-1}%
  \endgroup
}

%%%%%%%%%%%%%%%%%%%%
% TYPESET PACKAGES %
%%%%%%%%%%%%%%%%%%%%

% \usepackage[scaled,osf]{garamondx} %osf for text, lining for math
% \usepackage[scaled=.95]{cabin} %ss
% \usepackage[varqu,scaled=.95]{inconsolata} %tt
% \usepackage[garamondx,bigdelims,vvarbb]{newtxmath}
% \usepackage[cal=boondox]{mathalfa}
% \usepackage{textcomp}

% \let\circledS\undefined

% \renewcommand{\rmdefault}{phv} (Helvetica or altenative default font)

%\usepackage{bookman}
%\usepackage{charter}
%\usepackage{newcent}
%\usepackage{lmodern}
%\usepackage{mathpazo}

 \setkomafont{disposition}{\normalfont\bfseries} % Usual fonts for titles in Koma document classes

 %\setkomafont{sectioning}{\normalfont\bfseries}

%%%%%%%%%%%%%%%%%%%%%%%%%%%%%%%
% SPECIAL CHARACTERS PACKAGES %
%%%%%%%%%%%%%%%%%%%%%%%%%%%%%%%

%\usepackage{mathptmx}
%\usepackage{soul}
%\usepackage{ulem}
\usepackage{eurosym}
%\usepackage{marvosym}
%\usepackage{url}
%\usepackage{verbatim}
%\usepackage{moreverb}
%\usepackage{color}
%\usepackage{colortbl}
%\usepackage[framed,numbered,autolinebreaks,useliterate]{mcode}

\usepackage{lipsum}

\usepackage{amssymb}
\usepackage{amsmath}
\usepackage{amsthm}
\usepackage{dsfont}
\usepackage{stmaryrd}
\usepackage{mathtools} % Allows for use of dcases instead of cases
% \usepackage{apxproof} % For sending proofs to appendix

%\numberwithin{equation}{section}

% Maths objects
% Maths objects
\newtheorem{definition}{Definition}
\newtheorem{conjecture}{Conjecture}
\newtheorem{remark}{Remark}
\newtheorem{assumption}{Assumption}
\newtheorem{lemma}{Lemma}
%\newtheoremrep{lemma}{Lemma}
\newtheorem{theorem}{Theorem}
%\newtheoremrep{theorem}{Theorem}
\newtheorem{proposition}{Proposition}
%\newtheoremrep{proposition}{Proposition}
\newenvironment{sproof}{\renewcommand{\proofname}{Sketch of Proof}\proof}{\endproof}

\DeclareMathOperator*{\argmin}{arg\,min}
\DeclareMathOperator*{\argmax}{arg\,max}

\newcommand{\bigzero}{\makebox(0,0){\text{\huge0}}}
\newcommand{\prob}{\mathbb{P}}
\newcommand{\reals}{\mathbb{R}}
\newcommand{\Esp}{\mathbb{E}}
\newcommand{\Var}{\mathbb{V}}
\newcommand{\Cov}{\text{Cov}}
\newcommand{\naturals}{\mathbb{N}}
\newcommand{\rationals}{\mathbb{Q}}
\newcommand{\complex}{\mathbb{C}}
\newcommand{\Ftribe}{\mathcal{F}}
\newcommand{\Normal}{\mathcal{N}}
\newcommand{\Uniform}{\mathcal{U}}
\newcommand\bigp[1]{\bigl( #1 \bigr)}
\newcommand\eqnotebc[1]{\quad \text{(} \because \text{#1)}}
\newcommand{\ubar}{\overline{u}}
\newcommand{\cbar}{\overline{c}}
\newcommand{\pbar}{\overline{p}}
\newcommand{\Mcal}{\mathcal{M}}
\newcommand{\Pcal}{\mathcal{P}}
\newcommand{\indic}[1]{\mathds{1}_{ \{ #1 \} }}
\newcommand{\set}[1]{\{ #1 \}}


%%%%%%%%%%%%%%%%%%%%
% OBJECTS PACKAGES %
%%%%%%%%%%%%%%%%%%%%

%\usepackage{listings}

%\usepackage{makeidx}
%\usepackage{supertabular} % Tableaux (longs)
%\usepackage{hyperref} % Liens hypertextes
\usepackage{enumitem}

\usepackage{graphicx} % Images
\usepackage{wrapfig} % Figures
\usepackage[font=small, labelfont=sc]{caption} % Mise en forme des légendes
%\usepackage{placeins} % Allow FloatBarriers
\usepackage{subfigure}
\usepackage{pdfpages}

\usepackage{natbib}
\bibliographystyle{unsrtnat}

%%%%%%%%%%%%%%%%%%%%%%
% TITLE, AUTHORS,... %
%%%%%%%%%%%%%%%%%%%%%%

\title{Problem Set 2 \\ MA Math Camp 2021 }
\author{ Due Date : August 27th, 2021 }
\date{  }

\makeatletter
\let\thetitle\@title
\let\theauthor\@author
\let\thedate\@date
\makeatother

\newcommand\makesimpletitle{% 
\noindent 
\textbf{\large \thetitle} \\
\-\ \hspace{.2cm} { \large \theauthor } \\ 
\-\ \hspace{.2cm} { \normalsize \thedate }
}


%%%%%%%%%%%%%%%%%%%%%%%%%
%												%
%		DOCUMENT 						%
%						 						%
%%%%%%%%%%%%%%%%%%%%%%%%%



\begin{document}

%%% TITLEPAGE %%%

%%% MAIN MATTER %%%

\makesimpletitle

Answers should be typed and submitted in PDF format on Gradescope (see the course website for details). Be sure to answer every question thoroughly, and try to write complete and rigorous yet concise proofs. You can contact me if you have \emph{specific} questions about the problem set, or if you think you have spotted a typo or mistake.

\vspace{.5cm}

\begin{enumerate}
	
	\item Let $A$ a non-empty bounded subset of $\reals$. Show that $\inf A$ and $\sup A$ belong to the closure of $A$.

	\item Prove the following theorem. Let $\left(X,d_{X}\right)$, $\left(Y,d_{Y}\right)$, $\left(Z,d_{Z}\right)$ be metric spaces. Let set $S$ be a subset of $X$, and $f:S\to Y$ be a continuous function. Let $T$ be a set s.t. $f\left(S\right)\subset T\subset Y$, and $g:T\to Z$ be a continuous function. Then $g\circ f:S\to Z$ is a continuous function.

	\item Prove that the Euclidean space $\left(\mathbb{R}^{k},d_{2}\right)$ is a complete metric space. (Hint: First prove a Cauchy sequence in a metric space is bounded.)

	\item Check whether the following sets are subspaces of the $n$-dimensional real vector space $\mathbb{R}^{n}$, equipped with its usual addition and scalar product.
		\begin{enumerate}
		\item $\{\mathbf{0}\}$
		\item $\left\{ \mathbf{x} \in\mathbb{R}^{n}, \mathbf{x}=\alpha\mathbf{z},\text{ for some }\alpha\in\mathbb{R}\right\} $,where $\mathbf{z}\in\mathbb{R}^{n}$.
		\item $\{ (x_1,...,x_n) \in \reals^n, x_1 = 0 \}$
		\item $\{ (x_1,...,x_n) \in \reals^n, x_1 \neq 0 \}$
		\item $\{ (x_1,...,x_n) \in \reals^n, x_1 + x_2 = 0 \}$
		\item $\{ (x_1,...,x_n) \in \reals^n, x_1 = 0 \text{ or } x_2 = 0 \}$
		\item (When $n=1$) the set of integers $\mathbb{Z}$.
		\item (When $n=3$) $S:=\left\{ \left(t-2s,-s,t\right):t,s\in\mathbb{R}\right\} $.
		\item $KerA:=\left\{ \mathbf{v}\in\mathbb{R}^{n}:A\mathbf{v}=\mathbf{0}\right\} $,
		where $A$ is an $n\times n$ real matrix.
		\end{enumerate}

	\item Let $E$ a vector space and $F$ and $G$ two vector subspaces of $E$. Show that $F \cup G$ is a vector space if and only if $F \subset G$ or $G \subset F$. Show that $E$ cannot be written as the union of two vector subspaces different from $E$ itself.

	\item Consider the following collection of vectors in $\reals^4$ :
	\begin{align*}
	\begin{pmatrix}
	1 \\ 1 \\ 0 \\ 0
	\end{pmatrix},
	\begin{pmatrix}
	0 \\ 1 \\ 1 \\ 0
	\end{pmatrix},
	\begin{pmatrix}
	0 \\ 0 \\ 1 \\ 1
	\end{pmatrix},
	\begin{pmatrix}
	1 \\ 0 \\ 0 \\ 1
	\end{pmatrix},
	\end{align*}
	is it an independent family ?
	

	\item Show that the following $\Vert\cdot\Vert$ are valid norms in $\mathbb{R}^{n}$.
		\begin{enumerate}
		\item $\Vert\mathbf{x}\Vert:=\max_{i=1}^{n}\vert x_{i}\vert$.
		\item $\Vert\mathbf{x}\Vert:=\sum_{i=1}^{n}\vert x_{i}\vert$.
		\end{enumerate}

	\item Find non-zero $2\times2$ matrices, $A$, $B$ such that $AB=0$.

	\item Show that for any two $n \times n$ matrices $A$ and $B$, $Tr(AB)=Tr(BA)$.

	\item Let $A$ an $n \times n$ matrix and denote by $I_n$ the identity matrix of size $n$. Show that : there exists $\lambda \in \reals$ such that $A = \lambda I_n$ if and only if for any matrix $B$ of size $n$, $AB=BA$.
	

	\item Determine the rank of the following matrices:
	\begin{enumerate}
	\item $\left(\begin{array}{ccc}
	1 & 3 & 4\\
	2 & 0 & 1
	\end{array}\right)$
	\item $\left(\begin{array}{cccc}
	1 & 3 & 0 & 0\\
	2 & 4 & 0 & -1\\
	1 & -1 & 2 & 2
	\end{array}\right)$
	\item $\left(\begin{array}{cccc}
	1 & -2 & -1 & 1\\
	2 & 1 & 1 & 2\\
	-1 & 1 & -1 & -3\\
	-2 & -5 & -2 & 0
	\end{array}\right)$
	\end{enumerate}

	\item Is it possible that the vectors $\mathbf{v}_{1}$, $\mathbf{v}_{2}$,
	$\mathbf{v}_{3}$ are linearly dependent, but the vectors $\mathbf{v}_{1}+\mathbf{v}_{2}$,
	$\mathbf{v}_{1}+\mathbf{v}_{3}$, $\mathbf{v}_{2}+\mathbf{v}_{3}$
	are linearly independent?

	\item State whether each of the following statements is true or false. Justify it accordingly with a short proof or a counterexample.
		\begin{enumerate}
		\item No system of linear equations can have exactly $k$ solutions for
		any $k\geq2$.
		\item If $A\mathbf{x}=\mathbf{0}$ has a solution, then $A\mathbf{x}=\mathbf{b}$
		has a solution.
		\item If an $n\times n$ matrix $A$ is full rank, then $A\mathbf{x}=\mathbf{b}$
		has a solution.
		\item If an $n\times n$ matrix $A$ has rank less than $n$, then $A\mathbf{x}=\mathbf{b}$
		has no solution.
		\item If an $n\times n$ matrix $A$ is full rank, all its eigenvalues are
		distinct.
		\item Every diagonal real matrix has real eigenvalues.
		\item An $n\times n$ matrix $A$ has a zero eigenvalue if and only if it
		has rank less than $n$.
		\end{enumerate}

	\item Let $A$ be an $n\times n$ positive definite real matrix.
		\begin{enumerate}
		\item Verify that $\left\langle \cdot,\cdot\right\rangle :\mathbb{R}^{n}\times\mathbb{R}^{n}\to\mathbb{R}$
		such that
		\[
		\left\langle \mathbf{x},\mathbf{y}\right\rangle :=\mathbf{x}^{T}A\mathbf{y}
		\]
		is a valid inner product.
		\item Show that for any $\mathbf{x},\mathbf{y}\in\mathbb{R}^{n}$, we have
		$\left(\mathbf{x}^{T}A\mathbf{x}\right)\left(\mathbf{y}^{T}A\mathbf{y}\right)\geq\left(\mathbf{x}^{T}A\mathbf{y}\right)^{2}$. 
		\end{enumerate}

	\item Let $A$ be an idempotent matrix (i.e $A^2=A$). Show that the eigenvalues of $A$ must be either $0$ or $1$.

	\item Let $A$ a symmetric invertible $n \times n$ matrix. Show that $A^{-1}$ is symmetric.

\end{enumerate}

\end{document}
